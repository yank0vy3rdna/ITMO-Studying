\newpage
\rhead{\textbf{\textcolor{blue}{П}\textcolor{gray}{ример применения меры Харти на практике}}}
\makebox[0pt][l]{\includegraphics[scale=0.5]{img/pic.png} }
\vspace*{2mm}
\newline
\small{
\textbf{Пример 1.} Ведущий загадывает число от 1 до 64. Какое количество вопросов типа «да-нет» понадобится чтобы гарантированно угадать число?\\
\textbullet \ \underline{Первый} вопрос: «Загаданное число меньше 32?». Ответ: «Да». \\
\textbullet \ \underline{Второй} вопрос: «Загаданное число меньше 16?». Ответ: «Нет».\\
\textbullet \ $\ldots$\\
\textbullet \ \underline{Шестой} вопрос (в худшем случае) точно приведёт к верному ответу. \\
\textbullet \ Значит, в соответствии с мерой Хартли в загадке ведущего содержится ровно $log_2$64 = 6 бит непредсказуемости (т.е. информации).\\
\vspace*{2mm}
\textbf{Пример 2.} Ведущий держит за спиной ферзя и собирается поставить его на произвольную клетку доски. Насколько непредсказуемо его решение?\\
\textbullet \ Всего на доске 8х8 клеток, а цвет ферзя может быть белым или чёрным, т.е. всего возможно 8х8х2 = 128 равновероятных состояний.\\
\textbullet Значит, количество информации по Хартли равно $log_2$128 = 7 бит.\
}