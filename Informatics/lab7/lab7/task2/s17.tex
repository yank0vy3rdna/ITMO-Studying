\newpage
\rhead{\textbf{\textcolor{blue}{А}\textcolor{gray}{нализ свойств меры Хартли}}}
\makebox[0pt][l]{\includegraphics[scale=0.5]{img/pic.png} }
\vspace*{2mm}
\newline
Экспериментатор одновременно подбрасывает монету (М) и кидает игральную кость (К). Какое количество информации содержится в эксперименте (Э)?\\
\vspace*{2mm}
\textcolor{Green}{Аддитивность}:
\quad i(Э) = i(M) + i(K) => i(12 исходов) = i(2 исхода) + i(6 исходов): $log_x$12 = $log_x$2 + $log_x$6\\
\textcolor{Green}{Неотрицательность}:
\quad Функция $log_x$N неотрицательна при любом x>1 и N$\geqslant$1.\\
\textcolor{Green}{Монотонность}:
\quad С увеличением р(М) или р(К) функция i(Э) монотонно возрастает.\\
\textcolor{Green}{Принцип предопределённости}:
\quad При наличии всегда только одного исхода (монета и кость с магнитом) количество информации равно нулю: $log_x$1 + $log_x$1 = 0.