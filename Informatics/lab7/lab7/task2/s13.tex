\newpage
\rhead{\textbf{\textcolor{blue}{И}\textcolor{gray}{змерение количества информации}}}
\makebox[0pt][l]{\includegraphics[scale=0.5]{img/pic.png} }
\vspace*{2mm}
\newline
\textcolor{Green}{Количество информации $\equiv$ информационная энтропия - }
это численная мера непредсказуемости информации. Количество информации в некотором объекте определяется непредсказуемостью состояния, в котором находится этот объект.

\vspace*{1mm}
Пусть i(s) - функция для измерения количества информации в объекте s, состоящем из n независимых частей $s_k$, где k изменяется от 1 до n. Тогда
\textcolor{Green}{свойства меры количества информации} \textbf{i(s)} таковы:\\
\textbullet \ Неотрицательность: i(s)$\geqslant$0. \\
\textbullet \ Принцип предопределённости: если об объекте уже всё известно, то i(s)=0.\\
\textbullet \ Аддитивность: i(s)=$\sum$ i($s_k$) по всем k.\\
\textbullet \ Монотонность: i(s) монотонна при монотонном изменении вероятностей.